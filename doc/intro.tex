\chapter{Introduction}
\section{Purpose of Impetus}
Impetus (literal definition: \emph{the force that makes something
  happen or happen more quickly}) is an alarm clock that attempts to
reduce the possibility of the user to undermine the function of the
alarm clock.

An alarm clock's function is to wake the user up in order for them to
start their day. In many cases, however, the user can turn off the
clock in a way that requires so little effort and attention that they
can do so in a subconscious state. The goal of Impetus is to require
the user to be active in the deactivation of the alarm while also
limiting the complexity of that action.

\section{Approach}
There are roughly three states for Impetus: the idle state, the sleep
state, and the alarm state. In each state, the most important feature
used is the device's ability to detect human presence. Impetus uses an
infrared thermometer sensor to detect the presence of the user. This
in addition to scheduling features that will be discussed later make
Impetus controllable without physical input.

\subsection{Presence Detection}
For the purposes of this device, detecting a presence is done by
comparing the temperature of a distant object with the ambient
temperature of the sensor itself. If there is a large enough
difference (with the object being hotter), the device assumes someone
(or something) is present. How these readings are used vary according
to the current state of the device.

\subsubsection{Idle State}
In the idle state, the task of Impetus is to know at which time to
transition to the sleep state. Impetus is naive -- it treats any long
period of presence detection as good enough to transition.

\subsubsection{Sleep State}
In the sleep state, Impetus uses temperature readings (as well as
other statistics) to generate graphs to present to the user. These
metrics are meant to give some insight into sleeping habits, which
will be mentioned later. These metrics are sent to the user through
the Twitter social messaging service.

\subsubsection{Alarm State}
As the alarm goes off, if the temperature of the object in front of
the sensor is sufficiently above the ambient temperature of the sensor
(which is generally room temperature), then the clock will not stop
the alarm sequence. Only when the temperature sensed is near the
ambient temperature for a set, consecutive amount of time will the
alarm state complete. If at any time Impetus detects a presence, the
state will reset.

\subsection{Alarm Scheduling}
Scheduling is done through the Twitter. Requests modify the current
state of alarms (daily and impromptu alarms), which are saved between
sessions as well.

\section{Justification}
\subsection{Temperature Sensing}
Human presence through a contactless IR temperature sensor was chosen
as the main method of determining whether or not the alarm should turn
off because of what the desired result of an alarm should be -- the
user should not be anywhere near the alarm clock as it goes off. An
alarm has failed if that does not happen. If placed correctly, it
should be hard to cheat as well.

Other metrics were considered, like motion detection and range
detection, but were found to be missing key pieces of information to
make it work for an alarm.

\subsection{Online Scheduling and Twitter}
Online scheduling was chosen to completely get rid of physical
interaction with the device (in tandem with presence detection). As
something that is used on a daily basis in rarely changing patterns,
little interaction should be necessary to use an alarm clock.

Twitter was chosen as both a source for scheduling and a receiver for
statistics mainly to simplify the software side of Impetus. Scheduling
commands are small (Twitter messages are small), and images are easy
to consume on Twitter (graphs of metrics can be saved as images), so
while Twitter might not be the most sensible choice for these
operations, its nature fits both in some way.
