\chapter[Software]{Software\footnote{All the software written for this
  device -- and the \LaTeX~source for this technical manual -- is
  maintained at \texttt{https://github.com/seandw/impetus-clock} and
  is licensed under the MIT License.}}

\section{Arduino Pro Mini}
The Arduino Pro Mini's software only deals with communication between
sensors and actuators as prescribed by the current state, which it
maintains in conjunction with the Raspberry Pi. Its software depends
on the Arduino standard library, an I\textsuperscript{2}C library, and
a library that facilitates the creation of serial connections on
digital pins.

In addition to the pseudocode for \verb|setup()| and \verb|loop()|
below (which in itself ignores helper functions for communicating to
sensors), the Arduino Pro Mini code utilizes the Arduino construct
\verb|serialEvent()|, which allows incoming serial data to be read as
it comes in (after the current \verb|loop()| completes).

\subsection{Pseudocode}
\begin{verbatim}
setup():
  Setup serial communication with RPi.
  Setup software serial communication with display.
  Setup I2C communication with temperature sensor.
  Read ambient temperature, display on display.

serialEvent():
  while the RPi has more to send:
    Read byte of data.
    If data is the alarm command:
      Change state to the alarm state.
    Else if data is reset command:
      Change state to the idle state.
    Else:
      Write data to the display.

loop():
  If in sleep state:
    Read object temperature.
    Get current volume.
    Compare to values to two stored ones, keep the largest.
    If this has happened 60 times:
      Send stored temperature, volume, ambient temperature to RPi.
      Reset state variables.
  Else:
    If in alarm state and temperature is below ambient threshold:
      Increment count.
    Else if in idle state and temperature is above ambient threshold:
      Increment count.
    Else:
      Reset count.
      Allow buzzer to buzz if in alarm state.
    If in alarm state and count is above 10:
      Stop allowing the buzzer to buzz.
    If count is 60:
      If in alarm state:
        Switch to idle state.
      Else:
        Switch to sleep state.
        Dim display brightness.
        Notify RPi that the current state is sleep.

  If in alarm state:
    Buzz.

  Toggle colon on display, signifying a second has passed.
  Set points on display to signify current state.

  Wait a second.
\end{verbatim}

\section{Raspberry Pi}
The software for the Raspberry Pi is divided into three processes that
communicate between one another and the Arduino Pro Mini. The parent
process is responsible for not only spawning the other two processes,
but also to maintain state with the Arduino Pro Mini, receive sensor
data, generate graphs, and transmit it to the user via Twitter.

The \texttt{updateTime} child process is responsible for maintaining
the time on the Arduino Pro Mini. This process shares a serial
connection with the Arduino Pro Mini with its parent process, which
necessitates a synchronization lock.

The \texttt{updateSched} child process is responsible for maintaining
the alarm schedule and other configuration data, and frequently checks
the device's Twitter direct messages from the user to update the alarm
schedule. This process needs to communicate both configuration data
and alarm information to the parent process, which it does through a
pipe connection to the parent process.

Outside of the standard Python libraries, this software relies on the
\verb|pySerial| library for the serial connection to the Arduino Pro
Mini, the \verb|Twython| library for simplifying authentication and
calls to Twitter's REST API, and the \verb|matplotlib| library to
generate plots from sets of sensor data.

\subsection{Parent Process Pseudocode}
\begin{verbatim}
main():
  Open serial connection to Arduino Pro Mini
  Create updateTime process (as a daemon), lock.
  Create pipe, updateSched process.

  Receive Twitter configuration from pipe.

  Until interrupted:
    Flush all input from Arduino Pro Mini.
    Wait for signal from Arduino Pro Mini.
    Send current time through pipe.
    Until the alarm time the pipe sends back:
      Collect samples of temperature and volume from Arduino Pro Mini.
    Generate plot and tweet it to user.

  Lock and write reset code to Arduino Pro Mini.
  Close serial connection.

  Send teardown code through pipe.
  Close pipe, and join the updateSched process.
\end{verbatim}

\subsection{\texttt{updateTime} Process Pseudocode}
\begin{verbatim}
updateTime(start_time, serial, lock):
  Lock and write start_time to the Arduino Pro Mini

  Until killed:
    Lock and write current time to the Arduino Pro Mini.
    Sleep for 50 seconds.
\end{verbatim}

\subsection{\texttt{updateSched} Process Pseudocode}
\begin{verbatim}
updateSched(pipe):
  Open configuration file.
  Send Twitter configuration through pipe.

  Get last time direct messages were checked and the last id seen.

  Until interrupted:
    If there's data on the pipe:
      Receive data.
      If data is a time:
        Send the corresponding alarm time through the pipe.
      Else if it's the teardown code:
        Interrupt.
    If it's been 3 minutes since last DM check:
      Get new DMs.
      For each DM, modify alarm schedule according to the DM command.
      Store new id and time.

  Save configuration state.
  Close pipe connection.
\end{verbatim}
